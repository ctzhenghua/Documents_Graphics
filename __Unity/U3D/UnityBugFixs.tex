\documentclass[UTF8,a4paper,12pt]{ctexart} 

\usepackage{graphicx}%学习插入图
\usepackage{verbatim}%学习注释多行
\usepackage{booktabs}%表格
\usepackage{geometry}%图片
\usepackage{amsmath}
\usepackage{amssymb}
\usepackage{listings}%代码
\usepackage{xcolor}  %颜色
\usepackage{enumitem}%列表格式
\setenumerate[1]{itemsep=0pt,partopsep=0pt,parsep=\parskip,topsep=5pt}
\setitemize[1]{itemsep=0pt,partopsep=0pt,parsep=\parskip,topsep=5pt}
\setdescription{itemsep=0pt,partopsep=0pt,parsep=\parskip,topsep=5pt}
\usepackage{tcolorbox}
\usepackage{algorithm}  %format of the algorithm
\usepackage{algorithmic}%format of the algorithm
\usepackage{multirow}   %multirow for format of table
\usepackage{tabularx} 	%表格排版格式控制
\usepackage{longtable}
\usepackage{array}	%表格排版格式控制
\usepackage{hyperref} %超链接 \url{URL}
\usepackage{tikz}
\usepackage{dirtree}


\usetikzlibrary{intersections,
	positioning,
	petri,
	backgrounds,
	fit,
	decorations.pathmorphing,
	arrows,
	arrows.meta,
	bending,
	calc,
	intersections,
	through,
	backgrounds,
	shapes.geometric,
	quotes,
	matrix,
	trees,
	shapes.symbols,
	graphs,
	math,
	patterns,
	external}
\CTEXsetup[format+={\flushleft}]{section}

%%%% 设置图片目录
\graphicspath{{figure/}}

%%%% 段落首行缩进两个字 %%%%
\makeatletter
\let\@afterindentfalse\@afterindenttrue
\@afterindenttrue
\makeatother
\setlength{\parindent}{2em}  %中文缩进两个汉字位

%%%% 下面的命令重定义页面边距,使其符合中文刊物习惯 %%%%
\addtolength{\topmargin}{-54pt}
\setlength{\oddsidemargin}{0.63cm}  % 3.17cm - 1 inch
\setlength{\evensidemargin}{\oddsidemargin}
\setlength{\textwidth}{14.66cm}
\setlength{\textheight}{24.00cm}    % 24.62

%%%% 下面的命令设置行间距与段落间距 %%%%
\linespread{1.4}
\setlength{\parskip}{0.5\baselineskip}
\geometry{left=1.6cm,right=1.8cm,top=2cm,bottom=1.7cm} %设置文章宽度
\pagestyle{plain} 		  %设置页面布局

%代码效果定义
\definecolor{mygreen}{rgb}{0,0.6,0}
\definecolor{mygray}{rgb}{0.5,0.5,0.5}
\definecolor{mymauve}{rgb}{0.58,0,0.82}
\lstset{ %
	backgroundcolor=\color{white},   % choose the background color
	basicstyle=\footnotesize\ttfamily,      % size of fonts used for the code
	%stringstyle=\color{codepurple},
	%basicstyle=\footnotesize,
	%breakatwhitespace=false,         
	%breaklines=true,                 
	%captionpos=b,                    
	%keepspaces=true,                 
	%numbers=left,                    
	%numbersep=5pt,                  
	%showspaces=false,                
	%showstringspaces=false,
	%showtabs=false,        
	columns=fullflexible,
	breaklines=true,                 % automatic line breaking only at whitespace
	captionpos=b,                    % sets the caption-position to bottom
	tabsize=4,
	commentstyle=\color{mygreen},    % comment style
	escapeinside={\%*}{*)},          % if you want to add LaTeX within your code
	keywordstyle=\color{blue},       % keyword style
	stringstyle=\color{mymauve}\ttfamily,     % string literal style
	frame=L,
	xleftmargin = .079\textwidth,
	rulesepcolor=\color{red!20!green!20!blue!20},
	% identifierstyle=\color{red},
	language=c++,
}
 \title{\heiti U3D BugFix 统计表}
 
\begin{document}          %正文排版开始
 	\maketitle

		\begin{longtable}{p{2cm}|p{14cm}}
			\toprule
				类别 &  修复问题及版本	\\
			\midrule
				\verb|2D| &  
											\begin{itemize}
												\item 修复了在不同机器上构建图集时,存储图集哈希值的更改
												\item Fixed an issue where Sprite Atlases are modified after building
												\item Fixed a crash when selecting SpriteRenderer with tiling enabled in SceneView
												\item Fixed an exception when accessing ISpriteEditorDataProvider during AssetPostProcessor on new asset
											\end{itemize}	\\
				\hline
				\verb|UI| & 
											\begin{itemize}
											\item 修复了CanvasRenderer不遵守当前排序层/同级顺序的问题。
											\item Fixed an issue where EventSystem not being created when an EventSystem prefab was loaded but not in the scene
											\item Fixed an issue where Nested Canvases using the wrong display id when calculating the ProjectionMatrix. 
											\item Fixed an issue with IME properly. 
											\item Fixed a crash that occurred when importing prefabs with broken Canvas
											\end{itemize}			\\
				\hline
				\textbf{Asset 
				Pipeline} & 	
											\begin{itemize}
											\item 修复了重新序列化对嵌套式预制件进行更改的问题
											\item Fixed an assert thrown during RecompressAssetBundleAsync.
											\item Fixed an issue where asset bundles could fail to load when using async loading methods.
											\item Fixed an issue where calling SaveAssets() in play mode could cause an exception to be thrown.
											\item Fixed an issue where AssetBundle Recompression not verifying the CRC of bundle contents when requested.
											\item Optimised copying an asset, so that a refresh of the asset database is not triggered.
											\end{itemize}			\\
				\hline
				\verb|Editor| & 	
											\begin{itemize}
											\item 增加了对在批处理模式下执行异步方法的支持
											\item 修复了“ stackTraceLogType”命令行参数。
											\item 修复了2018.3。中引入的测试运行器api结果报告中的性能回归问题。
											\item 修复了崩溃报告将不可用的错误。
											\item 修复了针对相同纹理两次调用RenderTexture.ReleaseTemporary时崩溃的问题。
											\item 修复了使用空参数调用TextureImporter.ReadTextureSettings时崩溃的问题。
											\item 修复了当项目包含具有定义约束的dll资产时,切换目标平台时编辑器崩溃的问题。
											\item 修复了如果Profiler无法连接至Player时冻结编辑器的问题。
											\item 修复了UnityEngine.MinAttribute在使用Inspector窗口时不限制变量值的问题。
											\item Fixec creating an empty array in EditorBuildSettings.scenes crashing the Editor
											\end{itemize}			\\
				\hline
				\verb|Graphics| & 	
											\begin{itemize}
											\item 修复了启用保持四边形的网格无法渲染的问题。
											\item 修复了当网格仍具有索引但缺少顶点数据时动态批处理崩溃的问题。
											\item 修复了Metal Editor支持中的精灵伪像。
											\item Fixed an issue where per-platform QualitySettings could be stripped on disk when entering Playmode
											\item Fixed for GPU memory leak when deleting a texture while async upload in progress
											\item Fixed performance regression in Editor when processing shaders with errors.
											\end{itemize}			\\
				\hline
				\verb|Animation |  &	
											\begin{itemize}
											\item Fixed editor crash with WalkTypeTree when enforcing T-Pose to avatar after clearing all bones mapping
											\item 
											\end{itemize}			\\											
				\hline
				\verb|IL2CPP| & 			
											\begin{itemize}
											\item  修复了使用2017或更早版本的Unity构建的资产捆绑包中的着色器损坏的问题。
											\item 
											\end{itemize}			\\											
				\hline
				\verb|Physics| & 		
											\begin{itemize}
											\item 修复了当与触发器重叠的MeshCollider将其sharedMesh设置为null然后立即销毁时发生的崩溃。
											\item 修复了Cloth的问题,在Cloth中,更改SkinnedMeshRenderer的Mesh会导致编辑器崩溃。
											\end{itemize}			\\
				\hline
				\verb|Scripting| & 			
											\begin{itemize}
											\item 日志记录方法中的GC分配提高了50%。
											\item 修复了域重新加载后调用System.Diagnostics.Process API时崩溃的问题。
											\end{itemize}			\\
				\hline
				\verb|Profile| & 			
											\begin{itemize}
											\item 修复了在探查器窗口中显示UI详细信息时出现控制台错误的问题
											\item 修复了CPU事件探查器在进入PlayMode时将其视图类型更改回时间轴的问题。
											\item 固定的探查器窗口在所有图表均关闭时重新打开到“ CPU探查器详细信息”窗格。
											\end{itemize}			\\											
				\hline
				\verb|Shaders| & 			
											\begin{itemize}
											\item  修复了使用2017或更早版本的Unity构建的资产捆绑包中的着色器损坏的问题。
											\item 
											\end{itemize}			\\
				\hline
				\verb|Timeline| & 		
											\begin{itemize}
											\item 修复了在播放模式抛出异常期间复制控制剪辑的问题。
											\item 
											\end{itemize}			\\
				\hline
				\verb|iOS| & 		
											\begin{itemize}
											\item 现在,“编辑器”和Xcode项目中的“自动签名”设置处于“同步”状态,而“编辑器”中的默认“自动签名”设置现在设置为False。
											\item 固定的加速度计并未完全从陀螺仪上解耦。
											\item 修复了iPhone 11 pro上显示的错误dpi
											\item 
											\end{itemize}			\\											
				\hline
				\verb|Windows| &	
											\begin{itemize}
											\item 修复了未从Proxy中排除的“ localhost”导致编辑器组件失败的问题。
											\item 多重显示:固定在窗口模式下的非主要显示纵横比。
											\end{itemize}			\\
				\hline
				\verb|Android| & 	
											\begin{itemize}
												\item 修复了非全屏模式和拆分视图下的Android抠图
												\item 修复了POSIX平台上的FatalError和AccessViolation ForcedCrashCategory的功能。
												\item Fixed application paths (dataPath, streamingAssetsPath) to point to base apk when App Bundle is used.
												\item Set opaque windows for unset translucency attribute. 
											\end{itemize}		\\											
				\hline
				\verb|XR |  &	
											\begin{itemize}
											\item 修复了全息仿真窗口远程处理错误,该错误导致在选择Hololens 2时导致Unity远程访问Hololens 1设备。
											\item 修复了从UWP x64应用程序远程处理到V2设备的问题
											\end{itemize}			\\
																																				
			\bottomrule
		\end{longtable}

\end{document}
