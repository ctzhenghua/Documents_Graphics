 \documentclass[UTF8,a4paper,12pt]{ctexbook} 

 \usepackage{graphicx}%学习插入图
 \usepackage{verbatim}%学习注释多行
 \usepackage{booktabs}%表格
 \usepackage{geometry}%图片
 \usepackage{amsmath}
 \usepackage{amssymb}
 \usepackage{listings}%代码
 \usepackage{xcolor}  %颜色
 \usepackage{enumitem}%列表格式
 \usepackage{tcolorbox}
 \usepackage{algorithm}  %format of the algorithm
 \usepackage{algorithmic}%format of the algorithm
 \usepackage{multirow}   %multirow for format of table
 \usepackage{tabularx} 	%表格排版格式控制
 \usepackage{array}	%表格排版格式控制
 \usepackage{hyperref} %超链接 \url{URL}
\usepackage{dirtree}
 \CTEXsetup[format+={\flushleft}]{section}

 %%%% 段落首行缩进两个字 %%%%
 \makeatletter
 \let\@afterindentfalse\@afterindenttrue
 \@afterindenttrue
 \makeatother
 \setlength{\parindent}{2em}  %中文缩进两个汉字位
 
 
 %%%% 下面的命令重定义页面边距,使其符合中文刊物习惯 %%%%
 \addtolength{\topmargin}{-54pt}
 \setlength{\oddsidemargin}{0.63cm}  % 3.17cm - 1 inch
 \setlength{\evensidemargin}{\oddsidemargin}
 \setlength{\textwidth}{14.66cm}
 \setlength{\textheight}{24.00cm}    % 24.62
 
 %%%% 下面的命令设置行间距与段落间距 %%%%
 \linespread{1.4}
 % \setlength{\parskip}{1ex}
 \setlength{\parskip}{0.5\baselineskip}
 
 %%%% 下面的命令定义图表、算法、公式 %%%%
 \newcommand{\EQ}[1]{$\textbf{EQ:}#1\ $}
 \newcommand{\ALGORITHM}[1]{$\textbf{Algorithm:}#1\ $}
 \newcommand{\Figure}[1]{$\textbf{Figure }#1\ $}
 
 %%%% 下面命令改变图表下标题的前缀 %%%%% 如:图-1、Fig-1
 \renewcommand{\figurename}{Fig}
 
 \geometry{left=1.6cm,right=1.8cm,top=2cm,bottom=1.7cm} %设置文章宽度
 
 \pagestyle{plain} 		  %设置页面布局

 %代码效果定义
 \definecolor{mygreen}{rgb}{0,0.6,0}
 \definecolor{mygray}{rgb}{0.5,0.5,0.5}
 \definecolor{mymauve}{rgb}{0.58,0,0.82}
 \lstset{ %
 	backgroundcolor=\color{white},   % choose the background color
 	basicstyle=\footnotesize\ttfamily,      % size of fonts used for the code
 	%stringstyle=\color{codepurple},
 	%basicstyle=\footnotesize,
 	%breakatwhitespace=false,         
 	%breaklines=true,                 
 	%captionpos=b,                    
 	%keepspaces=true,                 
 	%numbers=left,                    
 	%numbersep=5pt,                  
 	%showspaces=false,                
 	%showstringspaces=false,
 	%showtabs=false,        
 	columns=fullflexible,
 	breaklines=true,                 % automatic line breaking only at whitespace
 	captionpos=b,                    % sets the caption-position to bottom
 	tabsize=4,
 	commentstyle=\color{mygreen},    % comment style
 	escapeinside={\%*}{*)},          % if you want to add LaTeX within your code
 	keywordstyle=\color{blue},       % keyword style
 	stringstyle=\color{mymauve}\ttfamily,     % string literal style
 	frame=single,
 	rulesepcolor=\color{red!20!green!20!blue!20},
 	% identifierstyle=\color{red},
 	language=c++,
 }
 \author{\kaishu 郑华}
 \title{\heiti 图形学理论 笔记}
 
\begin{document}          %正文排版开始
 	\maketitle
 	\tableofcontents
 	
 
 
\part{基础知识}
\chapter{参考文献}
	\section{视频} 清华大学:\url{http://cg.cs.tsinghua.edu.cn/course/resource_main.htm}
	
	\section{书籍} 胡世民 计算机图形基础教程
	
	
\chapter{背景知识}
	\section{绪论}
		计算机图形学是利用计算机研究图形的表示、生成、处理、显示的学科。
		
		\subsection{应用与意义}
			\begin{itemize}
				\item 电影 :科幻大片的特效
				\item 游戏 :图形地理、人物控制
				\item 计算机仿真
				\item CAD/CAM :设计、制造
	 			\item 建筑
				\item 可视化(科学):化学结构、生物
			\end{itemize}	
	
		\subsection{研究内容}
			\begin{itemize}
				\item 图形硬件、图形标准、图形交互技术
				\item 光栅图形生成算法
				\item 曲线曲面造型、实体造型
				\item 真实感图形绘制、科学计算可视化、计算机动画、自然景物仿真
				\item 虚拟现实
			\end{itemize}
			
	\section{涉及概念}
		\paragraph{图形与图象}
			\subparagraph{图像}
				计算机内以位图(Bitmap)形式存在的灰度信息。
		
			\subparagraph{图形}
				含有几何属性、更强调场景的几何表示,是由场景的几何模型和景物的物理属性共同组成的。
				
				图形主要分为两种:
				\begin{itemize}[itemindent = 1em]
					\item 基于线条信息表示
					\item 明暗图(\textbf{Shading})
				\end{itemize}
		
		\paragraph{图形学与CAD}
		\paragraph{图形学与模式识别}
		\paragraph{图形学与视觉}

	\section{历史}
		\begin{itemize}
			\item 细分曲面
			\item 光栅化图形:区域填充、裁剪、消隐等基本图形概念(70年代)
			\begin{enumerate}
				\item 光反射模型 70年(真实感图形学)
				\item 漫反射模型+插值 71年(明暗处理)
				\item 简单光照模型 75年(Phong 模型)
				\item 光投射模型 80年,并给出光线跟踪算法范例(Whitted 模型)
				\item 辐射度方法 84年
			\end{enumerate}

		\end{itemize}
		
	\section{杂志与会议}
		\paragraph{会议}
		\begin{itemize}
			\item Siggraph 
			\item Eurograph
			\item Pacific Graphics
			\item Computer Graphics International.
		\end{itemize}
		
		\paragraph{杂志}
			\begin{itemize}
				\item ACM Transaction on Graphics
				\item IEEE Computer Graphics and Application
				\item IEEE Visualization and Computer Graphics
			\end{itemize}

\chapter{光栅化}
	



\chapter{真实感绘制及重要概念}
	目的是模拟真实物体的物理属性,包括物体的形状,光学性质,表面的纹理和粗燥程度,以及物体间的相对位置,遮挡关系等。
	
	\dirtree{%
		.1 图形学理论.
		.2 \textbf{光照模型}.
		.3 简单光照模型.
		.3 局部光照模型.
		.3 整体光照模型.
		.2 \textbf{绘制方法}.
		.3 光线追踪.
		.3 辐射度.
		.2 \textbf{加速算法}.
		.3 包围盒树、自适应八叉树等.
		.3 阴影算法、纹理合成.
	}

	\section{具体案例}
		\begin{itemize}
			\item 基于预计算的全局光照实时绘制
			\item 表面细节绘制与体纹理
			\item 头发的交互绘制与编辑
		\end{itemize}
	
	\section{色彩视觉}
	
	\section{图形和像素}
	
	\section{三角网格模型}
	
	\section{光照模型与明暗处理}
		
		
		
		
\chapter{光照}
	\section{色彩空间}
		 RGB 3通道 
		 
		 CMY 减色系统
		  					
	\section{图象与像素}
	
	
	\section{三角网格模型}
		obj
		
		法向量
	



\part{应用实践}

	  
\end{document} 
 		    