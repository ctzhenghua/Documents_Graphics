\documentclass[UTF8,a4paper,12pt]{ctexart} 

 \usepackage{graphicx}%学习插入图
 \usepackage{verbatim}%学习注释多行
 \usepackage{booktabs}%表格
 \usepackage{geometry}%图片
 \usepackage{amsmath}
 \usepackage{amssymb}
 \usepackage{listings}%代码
 \usepackage{xcolor}  %颜色
 \usepackage{enumitem}%列表格式
 \usepackage{tcolorbox}
 \usepackage{algorithm}  %format of the algorithm
 \usepackage{algorithmic}%format of the algorithm
 \usepackage{multirow}   %multirow for format of table
 \usepackage{tabularx} 	%表格排版格式控制
 \usepackage{array}	%表格排版格式控制
 \usepackage{hyperref} %超链接 \url{URL}
 \CTEXsetup[format+={\flushleft}]{section}
  %%%% 下面的命令添加新字体 %%%%%
  \newfontfamily\kaishu{STKaiti}          % 楷体 - 解决编译新字体需要更新 字体缓存造成的长时间等待
  \newfontfamily\heiti{SimHei}
 
 
 %%%% 下面的命令定义图表、算法、公式 %%%%
 \newcommand{\EQ}[1]{$\textbf{EQ:}#1\ $}
 \newcommand{\ALGORITHM}[1]{$\textbf{Algorithm:}#1\ $}
 \newcommand{\Figure}[1]{$\textbf{Figure }#1\ $}
 
 %%%% 下面命令改变图表下标题的前缀 %%%%% 如:图-1、Fig-1
 \renewcommand{\figurename}{Fig}
 
 \geometry{left=1.6cm,right=1.8cm,top=2cm,bottom=1.7cm} %设置文章宽度
 
 \pagestyle{plain} 		  %设置页面布局

 %代码效果定义
 
 \definecolor{codegreen}{rgb}{0,0.6,0}
 \definecolor{codegray}{rgb}{0.5,0.5,0.5}
 \definecolor{codepurple}{rgb}{0.58,0,0.82}
 \definecolor{backcolour}{rgb}{0.98,0.92,0.84}
 \definecolor{applegreen}{rgb}{0.55, 0.71, 0.0}
 
 \lstdefinestyle{mystyle}{
 	language = {c++},%lua 语言指定方法
 	backgroundcolor=\color{backcolour},   
 	commentstyle=\color{codegreen},
 	keywordstyle=\color{magenta},
 	numberstyle=\tiny\color{codegray},
 	stringstyle=\color{codepurple},
 	basicstyle=\footnotesize,
 	breakatwhitespace=false,         
 	breaklines=true,                 
 	captionpos=b,                    
 	keepspaces=true,                 
 	%numbers=left,                    
 	%numbersep=5pt,                  
 	showspaces=false,                
 	showstringspaces=false,
 	showtabs=false,                  
 	tabsize=4
 }
 \lstset{style=mystyle, escapeinside=``}
	
 \author{\kaishu **}
 \title{\heiti 图形学程序训练指南}
 
\begin{document}          %正文排版开始
 	\maketitle
  
\section{计算机图形学-理论与实践}
	\subsection{图元的表示}
		\begin{itemize}
			\item 点
			\item 线
			\item 面
		\end{itemize}
		
		当然在表示图形的时候涉及一个点与点之间的链接方式:
		\begin{itemize}
			\item Line
			\item Triangle
			\item Quad
		\end{itemize}
		
		实现简单的长方形表示,并分别实现线-面渲染,分别使用不同链接方式。
	\subsection{光照添加}
		\begin{itemize}
			\item  类型
			\item  角度
			\item  强度
			\item  反射系数
		\end{itemize}
		
		在上述实验基础上,添加光照,注意光照的具体参数。
	\subsection{纹理添加}
		\begin{itemize}
			\item  纹理坐标
			\item  点与纹理坐标的对应
		\end{itemize}
		
		在上述实验基础上,添加纹理。
		
	\subsection{4大变换}
	
		\begin{itemize}
			\item 世界变换
			\item 投影变换
			\item 旋转变换
			\item 视图变换
		\end{itemize}
		
		整体实现,参考网上博客。重理论
	\subsection{动画原理}

\section{应用}
	\subsection{Object 如何表示}
		顶点坐标,索引坐标。
		
		\begin{itemize}
			\item 顶点坐标:物体的实际坐标
			\item 索引坐标:物体点之间的链接方式。
		\end{itemize}
		
	\subsection{如何添加算法于Object}
		这就看具体算法操作点还是索引了.
		    
\end{document} 
 		    