\documentclass[UTF8,a4paper,12pt]{ctexbook} 

 \usepackage{graphicx}%学习插入图
 \usepackage{verbatim}%学习注释多行
 \usepackage{booktabs}%表格
 \usepackage{geometry}%图片
 \usepackage{amsmath}
 \usepackage{amssymb}
 \usepackage{listings}%代码
 \usepackage{xcolor}  %颜色
 \usepackage{enumitem}%列表格式
 \usepackage{tcolorbox}
 \usepackage{algorithm}  %format of the algorithm
 \usepackage{algorithmic}%format of the algorithm
 \usepackage{multirow}   %multirow for format of table
 \usepackage{tabularx} 	%表格排版格式控制
 \usepackage{array}	%表格排版格式控制
 \usepackage{hyperref} %超链接 \url{URL}

 \CTEXsetup[format+={\flushleft}]{section}
  %%%% 下面的命令添加新字体 %%%%%
 
 %%%% 下面的命令定义图表、算法、公式 %%%%
 \newcommand{\EQ}[1]{$\textbf{EQ:}#1\ $}
 \newcommand{\ALGORITHM}[1]{$\textbf{Algorithm:}#1\ $}
 \newcommand{\Figure}[1]{$\textbf{Figure }#1\ $}
 
 %%%% 下面命令改变图表下标题的前缀 %%%%% 如:图-1、Fig-1
 \renewcommand{\figurename}{Fig}
 
 \geometry{left=1.6cm,right=1.8cm,top=2cm,bottom=1.7cm} %设置文章宽度
 
 \pagestyle{plain} 		  %设置页面布局

 %代码效果定义
 \definecolor{mygreen}{rgb}{0,0.6,0}
 \definecolor{mygray}{rgb}{0.5,0.5,0.5}
 \definecolor{mymauve}{rgb}{0.58,0,0.82}
 \lstset{ %
 	backgroundcolor=\color{white},   % choose the background color
 	basicstyle=\footnotesize\ttfamily,      % size of fonts used for the code
 	%stringstyle=\color{codepurple},
 	%basicstyle=\footnotesize,
 	%breakatwhitespace=false,         
 	%breaklines=true,                 
 	%captionpos=b,                    
 	%keepspaces=true,                 
 	%numbers=left,                    
 	%numbersep=5pt,                  
 	%showspaces=false,                
 	%showstringspaces=false,
 	%showtabs=false,        
 	columns=fullflexible,
 	breaklines=true,                 % automatic line breaking only at whitespace
 	captionpos=b,                    % sets the caption-position to bottom
 	tabsize=4,
 	commentstyle=\color{mygreen},    % comment style
 	escapeinside={\%*}{*)},          % if you want to add LaTeX within your code
 	keywordstyle=\color{blue},       % keyword style
 	stringstyle=\color{mymauve}\ttfamily,     % string literal style
 	frame=single,
 	rulesepcolor=\color{red!20!green!20!blue!20},
 	% identifierstyle=\color{red},
 	language=c++,
 }
 \author{\kaishu 郑华}
 \title{\heiti PCL 学习笔记}
 
\begin{document}          %正文排版开始
 	\maketitle
	\tableofcontents
	
\chapter{安装}
	\section{参考} \url{http://blog.csdn.net/qq1647243511/article/details/55190221}
	
	\section{注意事项}
		\subsection{Debug 下请务必 添加debug 版本的库} 一般以\verb|-gd| 结尾 或 \verb|-d| 结尾,当然在一个程序中也不能出现一部分用64位一部分用32位的..
		\subsection{批量保存 某目录下的文件名} 命令行进入目录后,\verb|dir >>name.txt /b|
		\subsection{不仅要配置 VS的 lib和include} 还要配置环境变量中的\verb|PATH|
		\subsection{编译出现std::numlimit 的错误括号不匹配时} 需要修改模版.h 文件 最后效果如\verb|(std::..min)()|
\chapter{数据读取与保存}

\chapter{精简}
	在本小节中我们将学习如何使用体素化网格方法实现下采样,即减少点的数量,减少点云数据,并同时保持点云的形状特征,在提高配准、曲面重建、形状识别等算法速度中非常实用。\verb|PCL|实现的\verb|VoxelGrid|类通过输入的点云数据创建一个三维体素栅格(可把体素栅格想象为微小的空间三维立方体的集合),然后在每个体素(即,三维立方体)内,用体素中所有点的重心来近似显示体素中其他点,这样该体素就内所有点就用一个重心点最终表示,对于所有体素处理后得到过滤后的点云。这种方法比用体素中心来逼近的方法更慢,但它对于采样点对应曲面的表示更为准确。
	
	\url{http://www.pclcn.org/study/shownews.php?lang=cn&id=67}

\chapter{重构}

\chapter{参考教程}
	\subparagraph{博客} \url{http://www.cnblogs.com/bozhicheng/category/884331.html}
	
		\url{http://blog.csdn.net/xuezhisdc/article/details/51034272}
		
		\url{http://blog.csdn.net/zhazhiqiang/article/details/52495872}
	
	\subparagraph{滤波} \url{http://blog.csdn.net/u012736279/article/details/50539199}
	
	\subparagraph{Tutorials}\url{http://www.pointclouds.org/documentation/tutorials/}	    
\end{document} 
 		    