 \documentclass[UTF8,a4paper,12pt]{ctexbook} 

 \usepackage{graphicx}%学习插入图
 \usepackage{verbatim}%学习注释多行
 \usepackage{booktabs}%表格
 \usepackage{geometry}%图片
 \usepackage{amsmath}
 \usepackage{amssymb}
 \usepackage{listings}%代码
 \usepackage{xcolor}  %颜色
 \usepackage{enumitem}%列表格式
 \usepackage{tcolorbox}
 \usepackage{algorithm}  %format of the algorithm
 \usepackage{algorithmic}%format of the algorithm
 \usepackage{multirow}   %multirow for format of table
 \usepackage{tabularx} 	%表格排版格式控制
 \usepackage{array}	%表格排版格式控制
 \usepackage{hyperref} %超链接 \url{URL}
 % \setCJKmainfont{方正兰亭黑简体}  %中文字体设置
 % \setCJKsansfont{华康少女字体} %设置中文字体
 % \setCJKmonofont{华康少女字体} %设置中文字体
\usepackage{tikz}
\usepackage{dirtree}


\usetikzlibrary{intersections,
	positioning,
	petri,
	backgrounds,
	fit,
	decorations.pathmorphing,
	arrows,
	arrows.meta,
	bending,
	calc,
	intersections,
	through,
	backgrounds,
	shapes.geometric,
	quotes,
	matrix,
	trees,
	shapes.symbols,
	graphs,
	math,
	patterns,
	external}
\CTEXsetup[format+={\flushleft}]{section}

%%%% 设置图片目录
\setmainfont{Times New Roman}
\graphicspath{{figure/}}

 %%%% 段落首行缩进两个字 %%%%
 \makeatletter
 \let\@afterindentfalse\@afterindenttrue
 \@afterindenttrue
 \makeatother
 \setlength{\parindent}{2em}  %中文缩进两个汉字位
 
 
 %%%% 下面的命令重定义页面边距,使其符合中文刊物习惯 %%%%
 \addtolength{\topmargin}{-54pt}
 \setlength{\oddsidemargin}{0.63cm}  % 3.17cm - 1 inch
 \setlength{\evensidemargin}{\oddsidemargin}
 \setlength{\textwidth}{14.66cm}
 \setlength{\textheight}{24.00cm}    % 24.62
 
 %%%% 下面的命令设置行间距与段落间距 %%%%
 \linespread{1.4}
 % \setlength{\parskip}{1ex}
 \setlength{\parskip}{0.5\baselineskip}
 
 %%%% 下面的命令定义图表、算法、公式 %%%%
 \newcommand{\EQ}[1]{$\textbf{EQ:}#1\ $}
 \newcommand{\ALGORITHM}[1]{$\textbf{Algorithm:}#1\ $}
 \newcommand{\Figure}[1]{$\textbf{Figure }#1\ $}
 
 %%%% 下面命令改变图表下标题的前缀 %%%%% 如:图-1、Fig-1
 \renewcommand{\figurename}{Fig}
 
 \geometry{left=1.6cm,right=1.8cm,top=2cm,bottom=1.7cm} %设置文章宽度
 
 \pagestyle{plain} 		  %设置页面布局

 %代码效果定义
 \definecolor{mygreen}{rgb}{0,0.6,0}
 \definecolor{mygray}{rgb}{0.5,0.5,0.5}
 \definecolor{mymauve}{rgb}{0.58,0,0.82}
 \lstset{ %
 	backgroundcolor=\color{white},   % choose the background color
 	basicstyle=\footnotesize\ttfamily,      % size of fonts used for the code
 	%stringstyle=\color{codepurple},
 	%basicstyle=\footnotesize,
 	%breakatwhitespace=false,         
 	%breaklines=true,                 
 	%captionpos=b,                    
 	%keepspaces=true,                 
 	%numbers=left,                    
 	%numbersep=5pt,                  
 	%showspaces=false,                
 	%showstringspaces=false,
 	%showtabs=false,        
 	columns=fullflexible,
 	breaklines=true,                 % automatic line breaking only at whitespace
 	captionpos=b,                    % sets the caption-position to bottom
 	tabsize=4,
 	commentstyle=\color{mygreen},    % comment style
 	escapeinside={\%*}{*)},          % if you want to add LaTeX within your code
 	keywordstyle=\color{blue},       % keyword style
 	stringstyle=\color{mymauve}\ttfamily,     % string literal style
 	frame=single,
 	rulesepcolor=\color{red!20!green!20!blue!20},
 	% identifierstyle=\color{red},
 	language=c++,
 }
 \author{\kaishu 郑华}
 \title{\heiti Unreal 笔记}
 
\begin{document}          %正文排版开始
 	\maketitle
 	\tableofcontents
 	
 
 

\chapter{Visual studio And Unreal Engine4}
	\section{Intruduction And installing UE4}
		
	\section{Installing Visual Studio with C++}
		
	\section{Setting up your first Project}
		\subparagraph{Jump Function Implements}
						
		\subparagraph{Write Simple Code Update Jump times}
	
	\section{Logging}
		
			\verb|UE_LOG(xx, xx, TEXT())|	
	
	
\chapter{Classes in Unreal Engine4}
	\section{Introduction to Classes and Actors}
		\subparagraph{Actor}
			
		\subparagraph{Pawn}
			
		\subparagraph{*Actor Component}
			
		\subparagraph{Character}
		
		\subparagraph{World}
			
			
	\section{Creating new Classes}
		\subparagraph{Introduction}
			\begin{itemize}[itemindent = 2em]
				\item Uses macro \verb|UCLASS()| to expose classes to the Engine
				\item A \verb|UCLASS| is a C++ calss but the \verb|UCLASS| macro will add header files to allow the intergration of your calss into the UE4 editor properly
				\item With a class being \verb|UCLASS|, its construction and desconstruction must be handled and managed by UE4
				\item Therefore, we may not use the \verb|new/delete| or \verb|malloc/free| operations to construct or delete the objects of type \verb|UCLASS|
			\end{itemize}
			
		\subparagraph{process}
			\verb|->|
			
				\begin{itemize}[itemindent = 2em ]
					\item creating Class based on the UE4's Actor class
					
					\item investigate what a UCLASS is made of
					
					\item Make your class and its properties editable
				\end{itemize}
			
			
			\begin{enumerate}[itemindent = 2em]
				\item To Engine Editor Select \verb|-> Add New|
				\item Select \verb|-> New C++ Class|
				\item Choose Parent Class \verb|Actor Pawn  ActorComponent ...|
			\end{enumerate}
	
	
	
	\section{Making Our Actors Present in Game}	
		\subparagraph{Component}
			\begin{itemize}[itemindent = 2em]
				\item Actors without components will have no visual representation and no transforms!
			\end{itemize}
	
		\subparagraph{Process}	
			\begin{enumerate}[itemindent = 2em]
				\item Add UStaticMeshComponent* to our Actor(Cpp Object)
				\item Locate a Suitable Static Mesh(World Object)
				\item Assign a Static Mesh to the component in C++(Connect Cppobject with Worldobject)
			\end{enumerate}

		
		
	\section{Implementing actor functionalities}
		\subparagraph{Process}
			\begin{enumerate}[itemindent = 2em]
				\item Establishing the PillSapwner Functionality
				\item Setting up \textbf{what} to spawn
				\item Establishing \textbf{where} to spawn
			\end{enumerate}
	
	
	\section{Spawning actors}
	
	
	


\chapter{进阶}
	
	\section{Memory Management in Unreal Engine4}
	


	
\chapter{其他相关领域}
	\section{Artificial Intelligence}	
	
	\section{Visual Computing}	
	
	\section{Virtual Reality}
	
	\section{Robotics}
	
	\section{参考文献} 
	
		基础概念 \url{http://www.52vr.com/article-569-1.html}
		
		系统 \url{http://docs.manew.com/ue4/index.html}
	    
\end{document} 
 		    