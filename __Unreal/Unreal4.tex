\documentclass[UTF8,a4paper,12pt]{ctexbook} 

\usepackage{graphicx}%学习插入图
\usepackage{verbatim}%学习注释多行
\usepackage{booktabs}%表格
\usepackage{geometry}%图片
\usepackage{amsmath}
\usepackage{amssymb}
\usepackage{listings}%代码
\usepackage{xcolor}  %颜色
\usepackage{enumitem}%列表格式
\setenumerate[1]{itemsep=0pt,partopsep=0pt,parsep=\parskip,topsep=5pt}
\setitemize[1]{itemsep=0pt,partopsep=0pt,parsep=\parskip,topsep=5pt}
\setdescription{itemsep=0pt,partopsep=0pt,parsep=\parskip,topsep=5pt}
\usepackage{tcolorbox}
\usepackage{algorithm}  %format of the algorithm
\usepackage{algorithmic}%format of the algorithm
\usepackage{multirow}   %multirow for format of table
\usepackage{tabularx} 	%表格排版格式控制
\usepackage{longtable}
\usepackage{array}	%表格排版格式控制
\usepackage{hyperref} %超链接 \url{URL}
\usepackage{tikz}
\usepackage{dirtree}


\usetikzlibrary{intersections,
	positioning,
	petri,
	backgrounds,
	fit,
	decorations.pathmorphing,
	arrows,
	arrows.meta,
	bending,
	calc,
	intersections,
	through,
	backgrounds,
	shapes.geometric,
	quotes,
	matrix,
	trees,
	shapes.symbols,
	graphs,
	math,
	patterns,
	external}
\CTEXsetup[format+={\flushleft}]{section}

%%%% 设置图片目录
\graphicspath{{figure/}}

%%%% 段落首行缩进两个字 %%%%
\makeatletter
\let\@afterindentfalse\@afterindenttrue
\@afterindenttrue
\makeatother
\setlength{\parindent}{2em}  %中文缩进两个汉字位

%%%% 下面的命令重定义页面边距,使其符合中文刊物习惯 %%%%
\addtolength{\topmargin}{-54pt}
\setlength{\oddsidemargin}{0.63cm}  % 3.17cm - 1 inch
\setlength{\evensidemargin}{\oddsidemargin}
\setlength{\textwidth}{14.66cm}
\setlength{\textheight}{24.00cm}    % 24.62

%%%% 下面的命令设置行间距与段落间距 %%%%
\linespread{1.4}
\setlength{\parskip}{0.5\baselineskip}
\geometry{left=1.6cm,right=1.8cm,top=2cm,bottom=1.7cm} %设置文章宽度
\pagestyle{plain} 		  %设置页面布局

%代码效果定义
\definecolor{mygreen}{rgb}{0,0.6,0}
\definecolor{mygray}{rgb}{0.5,0.5,0.5}
\definecolor{mymauve}{rgb}{0.58,0,0.82}
\lstset{ %
	backgroundcolor=\color{white},   % choose the background color
	basicstyle=\footnotesize\ttfamily,      % size of fonts used for the code
	%stringstyle=\color{codepurple},
	%basicstyle=\footnotesize,
	%breakatwhitespace=false,         
	%breaklines=true,                 
	%captionpos=b,                    
	%keepspaces=true,                 
	%numbers=left,                    
	%numbersep=5pt,                  
	%showspaces=false,                
	%showstringspaces=false,
	%showtabs=false,        
	columns=fullflexible,
	breaklines=true,                 % automatic line breaking only at whitespace
	captionpos=b,                    % sets the caption-position to bottom
	tabsize=4,
	commentstyle=\color{mygreen},    % comment style
	escapeinside={\%*}{*)},          % if you want to add LaTeX within your code
	keywordstyle=\color{blue},       % keyword style
	stringstyle=\color{mymauve}\ttfamily,     % string literal style
	frame=L,
	xleftmargin = .079\textwidth,
	rulesepcolor=\color{red!20!green!20!blue!20},
	% identifierstyle=\color{red},
	language=c++,
}
 \author{\kaishu 郑华}
 \title{\heiti Unreal4 笔记}
 
\begin{document}          %正文排版开始
 	\maketitle
 	\tableofcontents


\chapter{GamePlay}
	\section{基础世界观}
		\begin{figure}[H]
			\centering
			\includegraphics[width=\linewidth]{Base}
			\caption{Gameplay Framwork}
		\end{figure}
		
		\begin{itemize}
			\item Game Instance: 从游戏开始到游戏结束,可保存跨场景数据等。
			\item Game Mode: 储存游戏中不经常被修改的数据
			\item Pawn Class:人物、AI 载体实体
			\item HUD Class:二维显示信息
			\item Player Controller Class; 代表玩家的控制:包括输入操作、AI控制等
			\item Player State Class: 玩家的状态信息
			\item Game State Class: 任务状态等
		\end{itemize}
		
	\section{基础元素与组件}
		\subsection{UObject}
			UObject类提供了以下功能:
			\begin{itemize}
				\item 垃圾收集
				\item 引用自动更新
				\item 反射
				\item 序列化
				\item 自动检测 默认变量的更改
				\item 自动变量初始化
				\item 和虚幻引擎编辑器的自动交互
				\item 运行时类型识别
				\item 网络复制
			\end{itemize}
		
		\subsection{Actor}
			Actor类在场景中拥有一个位置坐标和旋转量。
			
			Actor类拥有这样的能力:\textbf{它能够被挂载组件} 。
			
			
		\subsection{Pawn}
			现Pawn类提供了\textit{被“操作”的 特性}。
			
			它能够被一个Controller操纵。这个Controller可以是玩家,当然 也可以是AI(人工智能)。这就像是一个棋手,操作着这个棋子。
			
			这就是Pawn类,一个被操纵的兵或卒,一个一旦脱离棋手就无法 自主行动的、悲哀的肉体。
			
		\subsection{Controller}
			既然是灵魂,那么肉体就不唯一,因此灵魂(Controller) 可以通过 Possess/UnPossess来控制一个肉体,或者从一个肉体上离开。
		
		\subsection{Character}
			Character类代表一个角色,它继承自Pawn类。
			
			Character类提供了一个特殊的组件,Character Movement。这个组 件提供了一个基础的、基于胶囊体的角色移动功能。包括移动和跳跃, 以及如果你需要,还能扩展出更多,例如蹲伏和爬行。


	\section{控制}
		
	
	
	\section{迭代}
		
	
	
	
	\section{命名规则}
		常用的前缀如下:
		\begin{itemize}
			\item F 纯C++类
			\item U 继承自UObject,但不继承自Actor
			\item A 继承自Actor
			\item S Slate控件相关类
			\item H HitResult相关类
		\end{itemize}
		
		
	\section{对象}
		
		\subsection{生成}
			在标准C++中,一个类产生一个对象,被称为“实例化”。实例化对 象的方法是通过new关键字。 而在虚幻引擎中,这一个问题变得略微复杂。对于某些类型,我们 不得不通过调用某些函数来产生出对象。具体而言:
			
			\begin{itemize}
				\item 如果你的类是一个\textbf{纯C++类}型(F开头),你可以通过\textbf{new}来产生对 象。
				\item 如果你的类\textbf{继承自UObject}但不继承自Actor,你需要通过 \textbf{NewObject函数}来产生出对象。 \verb|NewObject<T>()|
				\item 如果你的类\textbf{继承自AActor},你需要通过\textbf{SpawnActor函数}来产生出对 象。\verb|GetWorld()-> SpawnActor<AYourActorClass>()|
				\item 如果你需要产生出一个\textbf{Slate类},需要使用\textbf{SNew函数}。
			\end{itemize}
		
		\subsection{获取}:
		
			1· 通过new 获取引用或者指针。
			
			2·间接的通过遍历对象获取。
			
			\begin{lstlisting}
	for(TActorIterator <AActor> Iterator(GetWorld()); Iterator; ++Iterator) 
	{ ...//do something }			
			\end{lstlisting}
		
		\subsection{销毁}
			\begin{itemize}
				\item 纯C++ 类:符合C++ 基本语法
				\item UObject类:UObject采用自动垃圾回收机制。当一个\textbf{类的成员} 变量包含\textit{指向UObject的对象},\textit{同时又带有UPROPERTY宏定义},那么 这个成员变量将会触发引用计数机制。 垃圾回收器会定期从根节点Root开始检查,当一个UObject没有被 别的任何UObject引用,就会被垃圾回收。你可以通过AddToRoot函数来 让一个UObject一直不被回收。
				\item Actor 类:可以通过\textbf{调用Destory函数}来请求销毁,这样的销毁意 味着将当前Actor从所属的世界中“摧毁”。但是对象对应\textit{内存的回收依然 是由系统决定}。
			\end{itemize}
		


\chapter{蓝图}
	\section{基本概念}
	
		蓝图类,类似于prefab,定义一种\textbf{基于事件}的预工作流。
		
		\subsection{分类}
			
			\begin{itemize}
				\item 类蓝图 Blueprint Class:是一种允许内容创建者轻松地基于现有游戏性类添加功能的资源。
				\item 纯数据蓝图 Data-Only Blueprint:是指仅包含\underline{代码(以节点图表的形式)}、\textbf{变量}及\textbf{从其父类继承的组件}的\textit{类蓝图}。
				\item 关卡蓝图 Level Blueprint:用作关卡范围的\textbf{全局事件}图。
			\end{itemize}
		
		\subsection{蓝图接口}
			蓝图接口(Blueprint Interface) 是一个或多个函数的集合 - 只有名称,没有实施-可以添加到其他蓝图中。
		
		\subsection{蓝图宏库}
			蓝图宏库(Blueprint Macro Library) 是一个容器,它包含一组 宏 或自包含的图表,这些图表可以 作为节点放置在其他蓝图中。它们可以节省时间,因为它们可以存储常用的节点序列, 包括执行和数据传输所需的输入和输出。
		
		\subsection{蓝图工具}
			\textbf{蓝图编辑器(Blueprint Editor) 中}的 \textbf{组件(Components) 窗口}允许您\underline{将组件添加到蓝图}。这提供了以下方法: 通过胶囊组件(CapsuleComponent)、盒体组件(BoxComponent)或球体组件(SphereComponent)添加碰撞几何体, 以静态网格体组件(StaticMeshComponent)或金属网格体组件(SkeletalMeshComponent)形式添加渲染几何体, 使用移动组件(MovementComponent)控制移动。还可以将组件(Components)列表中添加的组件 指定给实例变量,以便您在此蓝图或其他蓝图的图表中访问它们。
		
		
		\subsection{核心组件}
			\begin{itemize}
				\item 构造函数 
				\item 事件图表:事件驱动
				\item 函数
				\item 变量
			\end{itemize}
			
	\section{使用规则}
		基于事件。
		
		控制流、数据流。
		
		
	\section{调用C++}
		如何让蓝图能够调用我的 C++类中的函数呢?
		
		\begin{itemize}
			\item \verb|UPROPERTY宏|:注册一个变量到蓝图中\verb|UPROPERTY(BlueprintReadWrite,VisibleAnywhere,Category="Object")|
			\item \verb|UFUNCTION宏|:注册函数到蓝图中 \verb|UFUNCTION(BlueprintCallable,Category="Test")|
		\end{itemize}
	
		
\chapter{事件}


\chapter{引擎系统}
	\section{正则}
		\verb|#include"Regex.h"|
	
	\section{FPaths}:
		\begin{itemize}
			\item 具体路径类 ,如:\verb|FPaths::GameDir()|可以获取到游戏根目录。
			\item 工具类 ,如:\verb|FPaths::FileExists()|用于判断一个文件是否存在。
			\item 路径转换类 ,如:\verb|FPaths::ConvertRelativePathToFull()|用于将相对 路径转换为绝对路径。
		\end{itemize}
	
		
	\section{XML、Json}
		\verb|Include"Json.h"|
	
	\section{GConfig}
		写配置、读配置
		
		\begin{lstlisting}
	GConfig->SetString( TEXT("MySection"), TEXT("Name"), TEXT("李白"), FPaths::GameDir()/ "MyConfig.ini");
	
	FString Result; 
	GConfig->GetString( TEXT("MySection"), TEXT("Name"), Result, FPaths::GameDir() / "MyConfig.ini");
		\end{lstlisting}
	
	\section{File}
		虚幻引擎提供了与平台无关的文件读写与访问接口,即\verb| FPlatformFileManager|。
	
	\section{UE\_LOG}
		\subsection{查看log}:
			Log窗口(\verb|Window->DeveloperTools- >OutputLog|)
		
		\subsection{log 类型}:
			UE\_LOG宏输出Log,第一个参数为Log的分类(需要预先定义)。 第二个参数为类型,有Log、Warning、Error 三种类型。
			
			\textbf{自定义Category}:\verb|DEFINE_LOG_CATEGORY_STATIC(LogMyCategory,Warning,All);|
			
		\subsection{打印log}:
			\begin{lstlisting}
	UE_LOG(LogMy, Warning, TEXT("Hell World")); 
	UE_LOG(LogMy, Warning, TEXT("Show a String %s"),*FString("Hello")); 
	UE_LOG(LogMy, Warning, TEXT("Show a Int %d"),100);			
			\end{lstlisting}
			
	\section{string}
		“文字”类型其实是一组类型:\textit{FName,FText和 FString。}这\textbf{三种类型}可以互相转换。
		
		\subsection{FName}
			FName是\textit{无法被修改的字符串},大小写不敏感。
			
		\subsection{FText}
			FText表示一个\textit{“被显示的字符串”}。所有你希望“显示”的字符串都应 该是FText。因为FText提供了内置的本地化支持,也通过一张查找表来 支持运行时本地化。FText不提供任何的更改操作,对于被显示的字符 串来说,“修改”是一个非常不安全的操作。
			
		\subsection{FString}
			FString是\textit{唯一提供修改操作的字符串类}。同时也意味着FString的消 耗要高于FName和FText。
		
	\section{Images}
	
	
	\section{编译器相关}
	

\chapter{AI-行为树}
	\section{行为-基础元素}
	
		\subsection{流程控制}:
			\begin{itemize}
				\item \verb|Selector|: 选择器 IF
				
					Selector节点会 从左到右逐个执行下面的子树,如果有一个子树返回true,它就会返回 true,只有所有的子树均返回false,它才会返回false。这就类似于日常 生活中“几个方案都试一试”的概念。
				\item \verb|Sequence|: 顺序执行器 STATEMENT
				
					Sequence节点会按顺序执行自己的子树,只有当前子树返回true,才会去执行下一个子树,直到 全部执行完毕,才会向上一级返回true。任何一个子树返回了false,它 就会停止执行,返回false。类似于日常生活中“依次执行”的概念。
				\item \verb|Parallel|: 并行执行 THREAD
			\end{itemize}
		
		\subsection{装饰器}:对子树的返回结果进行处理的节点。	
		
		\subsection{执行节点}:执行节点必然是\textit{叶子节点,执行具体的任务},并在任务 执行一段时间后\textit{,根据任务执行成功与否,返回true或者false}。
	
	
	\section{行为-细节}
		\subsection{Selector}
			终止条件:当某一个执行成功返回True。
			
			优先级:
		
		\subsection{Sequence}
			
		
		\subsection{Parallel}
		
		

\chapter{AI-强化学习}	





\chapter{编译}
	\section{模块、UnrealBuildTool}
		虚幻引擎的源码目录,会发现其按照四大部分: Runtime、Development、Editor、Plugin来进行规划,而\textit{每个部分内部, 则包含了一个一个的小文件夹。每个文件夹即对应一个模块}。
		
		\begin{itemize}
			\item Public文件夹 
			\item Private文件夹 
			\item .build.cs文件
		\end{itemize}
	
	\section{编译问题}
	
	




\chapter{光照}
	
\chapter{寻路}
	
\chapter{UMG}
		
\chapter{物理}
		
\chapter{动画}
	
\chapter{粒子系统}
							
\chapter{资源管理}

\chapter{Editor 扩展}
		
\chapter{跨平台发布apk}

\chapter{Profile}
	
\chapter{UE4 优化}

\chapter{调试技巧}
	
	
	
	
	
\end{document} 
 		    